\PassOptionsToPackage{english,french,dutch}{babel}
\documentclass[twoside]{extreport}
\usepackage{inbo_report}
\depotnr{D/2022/3241/290}
\year{2023}
\reviewer{
Erika Van den Bergh
}
\reportnr{Rapporten van het Instituut voor Natuur- en Bosonderzoek 2022
(26)}
\corresponding{\href{mailto:gunther.vanryckegem@inbo.be}{\nolinkurl{gunther.vanryckegem@inbo.be}}}
\doi{10.21436/inbor.85928183}
\coverdescription{Benthosstaalname op het Groot Schoor van Hamme (nieuwe
ontpoldering) eind september 2022 (foto door Dimitri Buerms).}
\shortauthor{Van Ryckegem, G., Vanoverbeke, J., Van de Meutter, F.,
Vandevoorde, B., Mertens, W., Mertens, A., Van Braeckel, A., Smeekens,
V., Thibau, K., Bezdenjesnji, O., Buerms, D., De Regge, N., Hessel, K.,
Lefranc, C., Soors, J., Van Lierop, F.}
\citetitle{MONEOS - Geïntegreerd datarapport INBO: Toestand Zeeschelde
2021. Monitoringsoverzicht en 1ste lijnsrapportage Geomorfologie,
diversiteit Habitats en diversiteit Soorten}

\establishment{Brussel}

\flandersfont{}
\usepackage{reportfont}

\makeatletter
\AtBeginDocument{%
   \expandafter\renewcommand\expandafter\section\expandafter
     {\expandafter\@fb@secFB\section}%
   \newcommand\@fb@secFB{\FloatBarrier
   \gdef\@fb@afterHHook{\@fb@topbarrier \gdef\@fb@afterHHook{}}}
   \g@addto@macro\@afterheading{\@fb@afterHHook}
   \gdef\@fb@afterHHook{}
}
\makeatother

\makeatletter
\AtBeginDocument{%
   \expandafter\renewcommand\expandafter\subsection\expandafter
     {\expandafter\@fb@subsecFB\subsection}%
   \newcommand\@fb@subsecFB{\FloatBarrier
   \gdef\@fb@afterHHook{\@fb@topbarrier \gdef\@fb@afterHHook{}}}
   \g@addto@macro\@afterheading{\@fb@afterHHook}
   \gdef\@fb@afterHHook{}
}
\makeatother


\title{MONEOS - Geïntegreerd datarapport INBO: Toestand Zeeschelde 2021}
\subtitle{Monitoringsoverzicht en 1ste lijnsrapportage Geomorfologie,
diversiteit Habitats en diversiteit Soorten}
\author{
Gunther Van Ryckegem, Joost Vanoverbeke, Frank Van de Meutter, Bart
Vandevoorde, Wim Mertens, Amber Mertens, Alexander Van Braeckel, Vincent
Smeekens, Koen Thibau, Olja Bezdenjesnji, Dimitri Buerms, Nico De
Regge, Kenny Hessel, Charles Lefranc, Jan Soors, Frederik Van Lierop
}
\colofonauthor{
Gunther Van Ryckegem, Joost Vanoverbeke, Frank Van de Meutter, Bart
Vandevoorde, Wim Mertens, Amber Mertens, Alexander Van Braeckel, Vincent
Smeekens, Koen Thibau, Olja Bezdenjesnji, Dimitri Buerms, Nico De
Regge, Kenny Hessel, Charles Lefranc, Jan Soors, Frederik Van Lierop}



\colophon{1}


% Alter some LaTeX defaults for better treatment of figures:
% See p.105 of "TeX Unbound" for suggested values.
% See pp. 199-200 of Lamport's "LaTeX" book for details.
%   General parameters, for ALL pages:
\renewcommand{\topfraction}{0.9}	% max fraction of floats at top
\renewcommand{\bottomfraction}{0.8}	% max fraction of floats at bottom
%   Parameters for TEXT pages (not float pages):
\setcounter{topnumber}{2}
\setcounter{bottomnumber}{2}
\setcounter{totalnumber}{4}     % 2 may work better
\setcounter{dbltopnumber}{2}    % for 2-column pages
\renewcommand{\dbltopfraction}{0.9}	% fit big float above 2-col. text
\renewcommand{\textfraction}{0.07}	% allow minimal text w. figs
%   Parameters for FLOAT pages (not text pages):
\renewcommand{\floatpagefraction}{0.7}	% require fuller float pages
% N.B.: floatpagefraction MUST be less than topfraction !!
\renewcommand{\dblfloatpagefraction}{0.7}	% require fuller float pages

\begin{document}
\maketitle
\pagenumbering{arabic}


%starttoc

\clearpage

\phantomsection
\addcontentsline{toc}{chapter}{\contentsname}
\setcounter{tocdepth}{0}
\tableofcontents

  


\clearpage

%endtoc

\hypertarget{macrozouxf6benthos}{%
\chapter{Macrozoöbenthos}\label{macrozouxf6benthos}}

Fichenummer: S\_DS\_V\_002\_benthos

\textbf{Frank Van de Meutter}, Jan Soors, Dimitri Buerms, Charles
Lefranc, Olja Bezdenjesnji, Joram De Beukelaer

\hypertarget{inleiding}{%
\section{Inleiding}\label{inleiding}}

Een beschrijving van de historische benthosgegevens in de Zeeschelde
(1999, 2002, 2005) die verzameld zijn met het oog op een
systeemmonitoring, is te vinden in Speybroeck et al.~(2014). Sinds 2008
wordt jaarlijks op basis van een random stratified design benthos
bemonsterd. De gegevens van 2008 tot en met 2021 worden geleverd in een
Excel-bestand (benthos\_data2008-2021\_rapportage2023.xlsx) met volgende
werkbladen.

\begin{itemize}
\tightlist
\item
  macrobenthos -- densiteit en biomassa per staalnamelocatie
\item
  locaties -- de Lambert-coördinaten van de bemonsterde locaties
\end{itemize}

\hypertarget{materiaal-en-methode}{%
\section{Materiaal en methode}\label{materiaal-en-methode}}

\hypertarget{strategie}{%
\subsection{Strategie}\label{strategie}}

Sinds 2008 wordt een stratified random sampling design toegepast. Als
hoogste hiërarchisch niveau binnen de stratificatie worden de 7
waterlichamen genomen, zoals deze voor monitoring en beoordeling in de
context van de Kaderrichtlijn Water (KRW) worden onderscheiden. De
benaming verschilt echter van de vorige rapportages en refereert nu aan
de saliniteit en verblijftijd in de verschillende zones. In de
Oligohaliene zone wordt de Rupel echter apart beschouwd en ook de Dijle
en Zenne worden als aparte eenheden behandeld. Per waterlichaam wordt
vervolgens een opdeling gemaakt per fysiotoop, met de uitzondering dat
hoog slik en slik in het supralitoraal (potentiële pionierzone) samen
genomen worden. Dit resulteert in een gelijkmatige spreiding van de
staalnamelocaties. Als basis bij de randomisatie werd de fysiotopenkaart
van 2019 gebruikt. De fysiotoop per waterlichaam fungeert als kleinste
eenheid van informatie. De stalen van verschillende locaties binnen een
zelfde fysiotoop x waterlichaam worden als replica's voor dat fysiotoop
beschouwd. In de Zeeschelde en de Rupel werden de antropogene harde
zones (steenbestortingen) afzonderlijk onderscheiden. Ook werden twee
delen van waterlichamen afgescheiden omwille van de enigszins afwijkende
aard van hun habitats en fauna. De zone Zoet kort verblijf werd
opgedeeld in het traject Melle-Gentbrugge en traject Zwijnaarde tot
ringvaart (``tijarmen'') versus de rest van het KRW-waterlichaam,
terwijl de Dijle stroomaf van de Zennemonding (Zennegat) werd
onderscheiden van de rest van de Dijle. In de grafieken in de
data-exploratie worden deze echter samengevoegd. Tot en met 2017 werd
het volledige estuarium jaarlijks bemonsterd. Vanaf 2018 worden de
zijrivieren Dijle, Nete en Zenne slechts 3-jaarlijks bemonsterd. Een
nolledige staalname gebeurde in 2020; de volgende volledige staalname
zal gebeuren in 2023. Jaarlijks worden nieuwe random vastgelegde
staalnamelocaties gekozen binnen elk van de strata. In principe worden 5
locaties per stratum (combinatie van fysiotoop en waterlichaam)
bemonsterd. Dit aantal wordt aangepast in sommige gevallen in functie
van de relatieve en absolute areaalgrootte van de fysiotopen binnen de
waterlichamen. Hoewel tijdens het nemen van de stalen veel aandacht gaat
naar het zo volledig mogelijk uitvoeren van de vooropgezette design, kan
doorgaans een klein aantal stalen niet genomen worden door technische
problemen of onvoorziene omstandigheden (bijvoorbeeld grote ongekende
velden breuksteen subtidaal die de staalname onmogelijk maken). In 2021
werden uiteindelijk 208 stalen genomen. Een overzicht van de stalen per
stratum is weergegeven in Tabel \ref{tab:070-staalnamelocaties}.

\begin{table}

\caption{\label{tab:070-staalnamelocaties}Aantal stalen per stratum in 2021.}
\centering
\resizebox{\linewidth}{!}{
\begin{tabular}[t]{lrrrrrr}
\toprule
waterloop & laag intertidaal & middelhoog/hoog intertidaal & diep subtidaal & matig diep subtidaal & ondiep subtidaal & hard substraat\\
\midrule
Durme & 5 & 10 & 0 & 4 & 5 & 0\\
Oligohalien & 4 & 8 & 4 & 4 & 4 & 3\\
Rupel & 4 & 8 & 5 & 5 & 4 & 3\\
Saliniteitsgradient & 9 & 18 & 9 & 9 & 8 & 3\\
Zoet kort verblijf & 7 & 14 & 4 & 3 & 10 & 6\\
\addlinespace
Zoet lang verblijf & 4 & 8 & 4 & 4 & 5 & 3\\
NA & 0 & 0 & 0 & 0 & 0 & 0\\
\bottomrule
\end{tabular}}
\end{table}

\hypertarget{staalname}{%
\subsection{Staalname}\label{staalname}}

We onderscheiden twee soorten benthosstalen.

\textbf{basisstaal (BS)}: jaarlijks * intertidaal: 1 steekbuisstaal
(diameter: 4,5cm) tot op een diepte van 15cm * subtidaal: 1
steekbuisstaal uit een Reineck box-corer staal (diameter: 4,5cm) tot op
een diepte van 15cm (in het box-corer staal).

\textbf{Oligochaetenidentificatiestaal (OID)}: elke drie jaar (2014,
2017, 2020 \ldots) werd tot en met 2017 een aanvullend een tweede
benthosstaal genomen. Dit staal wordt genomen in functie van de
identificatie van oligochaeten (OID). Vanad 2020 gebeurt de determinatie
echter op de oligochaeten die verzameld werden in het basisstaal.
Wanneer de Oligochaeten apart getrieerd zijn voor determinatie noemen we
deze oligochaetenfractie wel opnieuw het OID staal. Het staal werd op
dezelfde manier genomen als het basisstaal. De OID gegevens voor
staalnamejaar 2020 waren te laat beschikbaar en worden daarom besproken
in deze rapportage.

Beide benthosstalen (BS, OID) worden gefixeerd (F-Solv 50\%). Bij elk
benthosstaal wordt jaarlijks ook een \textbf{sedimentstaal} genomen met
een sedimentcorer (diameter 2 cm zie ook hoofdstuk 6.2) tot 10 cm diepte
in het substraat (intertidaal) of in het box-corer sample (subtidaal).
Dit wordt vervolgens ter bewaring ingevroren.

\hypertarget{verwerking}{%
\subsection{Verwerking}\label{verwerking}}

Hieronder geven we de chronologie van handelingen bij de verwerking van
elk type staal.

\textbf{BS}

\begin{itemize}
\tightlist
\item
  spoelen en zeven over twee zeven met maaswijdtes 1 mm en 500 µm
  =\textgreater{} twee zeeffracties. Elke fractie ondergaat de hierna
  volgende stappen.
\item
  uitselecteren van fauna
\item
  determineren van alle individuen tot op het laagst mogelijke
  taxonomische niveau + tellen (maar de Oligochaeta worden als 1 taxon
  gerekend)
\item
  biomassabepaling = verassing (`loss on ignition'):

  \begin{itemize}
  \tightlist
  \item
    per taxon (= soort of een hoger niet nader te determineren
    taxonomisch niveau)
  \item
    drogen (12h bij 105°C) =\textgreater{} drooggewicht (DW)
  \item
    verassen (2h bij 550°C) =\textgreater{} asgewicht (AW)
  \item
    biomassa: asvrij drooggewicht AFDW = DW -- AW
  \end{itemize}
\end{itemize}

\textbf{OID}

\begin{itemize}
\tightlist
\item
  spoelen en zeven over twee zeven met maaswijdtes 1mm en 500µm
  =\textgreater{} 2 zeeffracties
\item
  uitselecteren van fauna
\item
  determineren van 25 individuen Oligochaeta per zeeffractie tot op het
  laagst mogelijke taxonomische niveau + tellen totaal aantal wormen in
  het staal
\item
  geen biomassabepaling per soort; totale oligochaetenbiomassa wordt
  bepaald in BS ! Dit staal dient dus enkel voor het determineren van
  oligochaeten! Het bepalen van de soortspecfieke biomassa en densiteit
  gebeurt door de totale biomassa Oligochaeta in het BS staal te
  alloceren aan de verschillende taxa volgens hun relatieve aantallen in
  het OID staal. Deze methode houdt geen rekening met soortspecifieke
  biomassa's en is dus benaderend.
\end{itemize}

\hypertarget{resultaten}{%
\section{Resultaten}\label{resultaten}}

We bespreken hieronder de verkennende analyses van de jaarlijkse
standaard monitoringsinspanning (BS stalen) die jaarlijks gerapporteerd
wordt.

\hypertarget{resultaten-macrozouxf6benthos-2021}{%
\subsection{Resultaten macrozoöbenthos
2021}\label{resultaten-macrozouxf6benthos-2021}}

\hypertarget{densiteit-en-biomassa}{%
\subsubsection{Densiteit en biomassa}\label{densiteit-en-biomassa}}

De densiteit van het macrozoöbenthos in het Zeeschelde estuarium is in
de recente periode relatief stabiel (Figuur
\ref{fig:070-figuur10-soorten}, \ref{fig:070-figuur11-soorten}). Omdat
deze parameter inherent grote fluctuaties ondergaat, wordt deze
beoordeeld op zijn logaritmisch verloop. De veranderingen in 2021 vallen
binnen de langjarige variatie. Na recordjaar 2019 in de zone Zoet kort
verblijf daalden de waarden twee jaar op rij, maar ze blijven wel de
hoogste in het Zeeschelde estuarium. Die hoge waarden komen volledig op
het conto van de kreekvormige aantakkingen (tijarmen) van
Gentbrugge-Melle en in mindere mate van de tijarm Zwijnaarde. De
zijrivieren vertoonden matig hoge densiteiten binnen de variatie van het
laatste decennium. De biomassa dichtheid (g per m²) van het
macrozoöbenthos in 2021 was vrij hoog in de zijrivieren, maar eerder
laag in de Zeeschelde, met de belangrijke uitzondering van de zone
Saliniteitsgradiënt. De opvallende stijging in de zone
Saliniteitsgradiënt (Zeeschelde IV, ongeveer Antwerpen tot de
Nederlandse grens) die startte in 2019 zette zich voor het derde
opeenvolgende jaar door. In de zone Zoet kort verblijf lijkt een einde
gekomen aan een periode van zeer hoge biomassa dichtheid sinds 2015. De
biomassa dichtheid viel terug tot waarden die we hier vóór 2015
noteerden. Het wordt interessant om op te volgen of dit om een
tijdelijke dip gaat of om een structurele trend. Het aandeel lege stalen
was opvallend laag voor de meeste zones, behalve in de oligohaliene zone
waar voor het tweede jaar op rij bijna 40\% lage stalen werden
vastgesteld (Figuur \ref{fig:070-figuur12-soorten}).

\begin{figure}[H]

{\centering \includegraphics[width=1\linewidth]{G:/.shortcut-targets-by-id\0B0xcP-eNvJ9dZDBwVVJOVk5Ld2s/PRJ_SCHELDE/VNSC/Rapportage_INBO/2023/070_macrozoobenthos/figuren/070-figuur-densiteitgemiddelde} 

}

\caption{Gemiddelde densiteit (lijn) aan macrozoöbenthos per waterlichaam opgedeeld in subtidaal en intertidaal. De spreiding rond de lijn wordt begrensd door het 1ste quartiel en 3de quartiel. }\label{fig:070-figuur10-soorten}
\end{figure}

\begin{figure}[H]

{\centering \includegraphics[width=1\linewidth]{G:/.shortcut-targets-by-id\0B0xcP-eNvJ9dZDBwVVJOVk5Ld2s/PRJ_SCHELDE/VNSC/Rapportage_INBO/2023/070_macrozoobenthos/figuren/070-figuur-biomassagemiddelde-waterlichaam-alternatief} 

}

\caption{Gemiddelde biomassa (lijn) aan macrozoöbenthos per waterlichaam opgedeeld in subtidaal en intertidaal. Met weergave van spreiding 1ste quartiel en 3de quartiel.}\label{fig:070-figuur11-soorten}
\end{figure}

\begin{figure}[H]

{\centering \includegraphics[width=0.7\linewidth]{G:/.shortcut-targets-by-id\0B0xcP-eNvJ9dZDBwVVJOVk5Ld2s/PRJ_SCHELDE/VNSC/Rapportage_INBO/2023/070_macrozoobenthos/figuren/070-figuur-aandeel-lege-stalen} 

}

\caption{Aandeel aan lege stalen per waterlichaam doorheen de tijd.}\label{fig:070-figuur12-soorten}
\end{figure}

Door de vastgestelde biomassa dichtheidswaarden (in g per m²) te
vermenigvuldigen met de aanwezige oppervlakte aan verschillende
fysiotopen, kunnen we de totale, in het systeem aanwezige, biomassa
macrozoöbenthos berekenen. Voor de berekening werden de
ecotoopoppervlaktes gebruikt van de jaargangen waarvoor gebiedsdekkende
ecotoopkaarten voorhanden waren (2010, 2013, 2016,2019). De oppervlaktes
uit kaartjaar 2010 werden gelinkt aan de benthosjaren vóór 2012. De
oppervlaktes uit kaartjaar 2013 werden gelinkt aan de benthosjaren
2012-2014, de oppervlaktes uit kaartjaar 2016 werden gelinkt aan de
benthosjaren 2015-2017 en de oppervlaktes uit kaartjaar 2019 werden
gelinkt aan de benthosjaren 2018-2020. Voor 2021 werden de
fysiotoopoppervlaktes voor 2021 gebruikt. De systeembiomassa (Figuur
\ref{fig:070-figuur13}) kende een dip in 2017 voor het stroomafwaartse
deel van het estuarium vanaf Durme, Rupel en de zone Saliniteitsgradiënt
waarna deze sterk opveerden in 2018 wat zich daarna doorzette tot en met
2021 en culimineerde in een nieuw maximum voor de Zeeschelde sinds de
start van de monitoring in 2008.

\begin{figure}[H]

{\centering \includegraphics[width=0.7\linewidth]{G:/.shortcut-targets-by-id\0B0xcP-eNvJ9dZDBwVVJOVk5Ld2s/PRJ_SCHELDE/VNSC/Rapportage_INBO/2023/070_macrozoobenthos/figuren/070-figuur-intertidalesysteembiomassa} 

}

\caption{Gesommeerd totaal van de gemiddelden per stratum van de systeembiomassa per waterlichaam en voor de totale Zeeschelde, uitgedrukt in ton asvrij drooggewicht.
Doelstelling systeemniveau is 30 ton; doelstellingen per deelzones zijn op de figuur weergegeven door een horizontale lijn met bij het waterlichaam passende kleur (Saliniteistgradiënt=14.2, Oligohalien=8.3, Zoet lang verblijf=5, Zoet kort verblijf=2.5).}\label{fig:070-figuur13}
\end{figure}

De aanhoudende hoge systeembiomassa van de Zeeschelde sinds 2017 was met
name in de periode 2019 tot en met 2021 volledig toe te schrijven aan de
zone Saliniteitsgradiënt. Aangezien er geen noemenswaardige
fysiotoopoppervlakteveranderingen gebeurden is de toename volledig toe
te schrijven aan een effectieve toename van macrozoöbenthosbiomassa in
dit deel van de Zeeschelde. Deze toename komt helemaal op het conto van
de Bivalvia (zie Figuur \ref{fig:070-figuur13b}). In de eerste plaats is
er de vestiging van een exotische nieuwkomer, de brakwaterkorfschelp
(\emph{Potamocorbula amurensis}) die in 2018 voor het eerst is
vastgesteld in de Zeeschelde (meteen de ook eerste vondst in Europa,
Dumoulin \& Langeraert, 2020). De soort breidt sindsdien sterk uit en
was vorig jaar (2020) al verantwoordelijk voor ongeveer 17\% van de
intertidale systeembiomassa macrozoöbenthos in de Zeeschelde. Maar ook
andere bivalven doen het goed, met hoge aantallen van het Nonnetje
(\emph{Limecola balthica}) in 2020, en vooral een hoge biomassa
dichtheid van de Platte slijkgaper in 2021 (\emph{Scrobicularia plana})
(zie Figuur \ref{fig:070-figuur13c}). Het is de eerste keer dat we deze
tweekleppige vaststellen tijdens de monitoring, maar we zien de
aantallen van deze in hoofdzaak mariene soort, die normaalgezien in zeer
lage aantallen gezien wordt, al enkele jaren fors toenemen in de
Zeeschelde. Een belangrijke opmerking bij de trends van de bivalven is
dat de gebruikte staalnamemethode met een erg smalle steekbuis niet
geschikt is om organismen van de grootte en de ruimtelijke distributie
van bivalven te bemonsteren. Toevalseffecten spelen dan een grote rol
waardoor trends minder stabiele zijn tussen jaren. Zo is de
brakwaterkorfschelp nog steeds in grote aantallen aanwezig, maar is ze
in 2021 nauwelijks vastgesteld in het intertidaal, maar wel nog in grote
aantallen in het subtidaal (zie Figuur \ref{fig:070-figuur13d}).
Bovendien vinden we de bivalven (uitgezonderd \emph{Dreissena sp}, maar
die zitten subtidaal op hard substraat en worden niet of enkel toevallig
bemonsterd) voornamelijk terug in de zone Saliniteitsgradiënt met
(veruit) de grootste fysiotoopoppervlaktes in de Zeeschelde, zodat de
toevalseffecten sterk uitvergroot worden.

De evaluatiegrenswaarde werd inn 2021 ook ruimschoots gehaald, doordat
de systeembiomassa verder steeg tot een totaal van 45.68 ton droge stof.
De verdeling van de biomassa over de verschillende zones van het
estuarium wijkt wel af van de doelstellingen (zie Figuur
\ref{fig:070-figuur13}). De zones Oligohalien en Zoet lang verblijf
zitten nog steeds onder de vooropgestelde zone-specifieke minimumgrens
(Figuur \ref{fig:070-figuur13}). Hoewel een erg hoge systeembiomassa
behaald werd in 2021, stelden we tegelijk een opvallend afname vast van
de systeembiomassa Oligochaeta. Omdat Oligochaeta klein zijn en vrij
heterogeen verspreid zitten is deze afname weinig aan toevalseffecten
onderhevig. De afname van Oligochaeta gebeurde in de zones zoet kort
verblijf en Oligohalien.

\textbackslash begin\{figure\}{[}H{]}

\{\centering \includegraphics[width=0.7\linewidth]{G:/.shortcut-targets-by-id\0B0xcP-eNvJ9dZDBwVVJOVk5Ld2s/PRJ_SCHELDE/VNSC/Rapportage_INBO/2023/070_macrozoobenthos/figuren/PopTaxgroep_ZS}

\}

\textbackslash caption\{Jaarlijkse systeembiommassa (ton droge stof)
voor de 7 belangrijkste Taxon groepen in de Zeeschelde. Deze groepen
bevatten jaarlijks samen meer dan 95\% van de systeembiomassa in de
Zeeschelde. De toename van de Bivalvia is vrijwel volledig te wijten aan
de de brakwaterkorfschelp \}\label{fig:070-figuur13b}
\textbackslash end\{figure\}

\textbackslash begin\{figure\}{[}H{]}

\{\centering \includegraphics[width=0.7\linewidth]{G:/.shortcut-targets-by-id\0B0xcP-eNvJ9dZDBwVVJOVk5Ld2s/PRJ_SCHELDE/VNSC/Rapportage_INBO/2023/070_macrozoobenthos/figuren/BIVsoortenINTER_ZS}

\}

\textbackslash caption\{Jaarlijkse systeembiommassa (ton droge stof)
voor de belangrijkste tweekleppigen (Bivalvia) in het intertidaal van de
Zeeschelde. Deze soorten bevatten jaarlijks samen meer dan 95\% van de
systeembiomassa in de Zeeschelde. \}\label{fig:070-figuur13c}
\textbackslash end\{figure\} \textbackslash begin\{figure\}{[}H{]}

\{\centering \includegraphics[width=0.7\linewidth]{G:/.shortcut-targets-by-id\0B0xcP-eNvJ9dZDBwVVJOVk5Ld2s/PRJ_SCHELDE/VNSC/Rapportage_INBO/2023/070_macrozoobenthos/figuren/BIVsoortenSUB_ZS}

\}

\textbackslash caption\{Jaarlijkse systeembiommassa (ton droge stof)
voor de belangrijkste tweekleppigen (Bivalvia) in het subtidaal van de
Zeeschelde. Deze soorten bevatten jaarlijks samen meer dan 95\% van de
systeembiomassa in de Zeeschelde.\}\label{fig:070-figuur13d}
\textbackslash end\{figure\}

\hypertarget{soortenrijkdom}{%
\subsubsection{Soortenrijkdom}\label{soortenrijkdom}}

Volledige determinatie aan de hand van BS- en OID-stalen gebeurt elke
drie jaar. Ook 2020 was een jaar met volledige determinatie inclusief
Oligochaeta, maar de resultaten zullen pas in de volgende rapportage
verschijnen. Een overzicht van de soortenrijkdom voor de verschillende
waterlichamen en de verschillende jaren per tidale zone (inter-, sub-)
staat in figuren \ref{fig:070-figuur15} en \ref{fig:070-figuur15b}. In
deze figuren zijn alle beschikbare determinaties opgenomen, waardoor de
weergegeven soortenrijkdom in de jaren met OID identificatie (zie hoger)
groter is. Vergelijken moet dus gebeuren tussen jaren met dezelfde
telmethode, maar omdat voor 2020 de Oligochaeta gegevens nog niet
inbegrepen zijn, moet deze (in deze rapportage) met de jaren met
onvolledige determinatie vergeleken worden. In vrij veel zones ligt de
soortenrijkdom de laatste 2 onderzoeksjaren iets hoger dan in de
voorbije periode. Mogelijke oorzaken zijn het steeds toenemend aantal
exotische soorten en de uitzonderlijke droogteperioden die mogelijks
marinisatie van de Zeeschelde veroorzaakten. Dit moet verder uitgeklaard
worden. Opvallend is dat enkel in het Oligohalien de taxa rijkdom hoger
is in het subtidaal dan in het intertidaal (Oligochaeta niet meegeteld).
Dit patroon is stabiel in de tijd maar een reden ervoor is niet gekend.

\begin{figure}[H]

{\centering \includegraphics[width=1\linewidth]{G:/.shortcut-targets-by-id\0B0xcP-eNvJ9dZDBwVVJOVk5Ld2s/PRJ_SCHELDE/VNSC/Rapportage_INBO/2023/070_macrozoobenthos/figuren/070-figuur-soortenrijkdom-Zeeschelde} 

}

\caption{Staalgemiddelde soortenrijkdom (boxplots; mediaan, IQrange) per waterlichaam doorheen de tijd. De Oligochaeta worden niet steeds gedetermineerd en werden als 1 taxon beschouwd, behalve in de jaren 2008, 2011, 2014, 2017 en 2020. De jaren onderling vergelijken kan dus enkel tussen deze opgesomde jaren, en tussen de tussenliggende jaren.}\label{fig:070-figuur15}
\end{figure}
\begin{figure}[H]

{\centering \includegraphics[width=1\linewidth]{G:/.shortcut-targets-by-id\0B0xcP-eNvJ9dZDBwVVJOVk5Ld2s/PRJ_SCHELDE/VNSC/Rapportage_INBO/2023/070_macrozoobenthos/figuren/070-figuur-soortenrijkdom-lijn-Zeeschelde} 

}

\caption{Soortenrijkdom per waterlichaam doorheen de tijd. De Oligochaeta worden niet steeds gedetermineerd en werden als 1 taxon beschouwd, behalve in de jaren 2008, 2011, 2014, 2017 en 2020. De jaren onderling vergelijken kan dus enkel voor deze opgesomde jaren, en de tussenliggende jaren. In de zijrivieren wordt sinds 2017 niet meer jaarlijks maar driejaarlijks bemonsterd tijdens de OID jaren, waardoor de diversiteit hoger lijkt.}\label{fig:070-figuur15b}
\end{figure}

\hypertarget{soortendiversiteit-shannon-index}{%
\subsubsection{Soortendiversiteit
Shannon-index}\label{soortendiversiteit-shannon-index}}

De Shannon diversiteit is een relatief nieuwe evaluatieparameter. Ze
wordt berekend op zowel biomassa (g droge stof/m²) als aantallen van het
macrozoöbenthos. De Oligochaeta werden over alle jaren als één taxon
beschouwd. We berekenen de Shannon diversiteit voor de vier KRW
onderdelen van de Zeeschelde (niveau 3) en voor de totale Zeeschelde. De
evolutie van deze parameter overheen de jaren per tidale zone (inter-,
sub-) staat in de figuren \ref{fig:070-figuur16} en
\ref{fig:070-figuur17}. De Shannon index voor het intertidaal in de
gehele Zeeschelde is vrij stabiel doorheen de tijd. De Zeeschelde als
geheel en de zone sterke Saliniteitsgradiënt hebben een duidelijk hogere
Shannon diversiteit dan de overige zones, die vrij laag scoren.
Opmerkelijk is dat we voor de deelzones wel een algemene toenemende
trend zien, met name sinds het jaar 2015. Dit is vooral uitgesproken
voor densiteiten, maar minder voor biomassa, maar in 2021 is er wel een
terugval. Enkel voor de zone Zoet lang verblijf is er in de laatste 2
jaren weer een terugval. Voor het subtidaal zijn de patronen behoorlijk
erratisch, wat waarschijnlijk te wijten is aan de veel lagere aantallen
macrobenthos die we hier vinden waardoor de invloed van toeval op de
parameter relatief groter is. Met wat goede wil is ook hier een opvering
van de Shannon index in de deelgebieden merkbaar sinds 2015, maar
variatie tussen de jaren is groot. Een opmerkelijk patroon is te zien in
de zone sterke Saliniteitsgradiënt: bij densiteiten is er een toename,
terwijl er voor biomassa een sterke afname van de Shannon diversiteit is
in de laatste 2 jaren. Waarschijnlijk is dit te wijten aan de opkomst
van de brakwaterkorfschelp (Dumoulin \& Langeraert, 2020).

\begin{figure}[H]

{\centering \includegraphics[width=1\linewidth]{G:/.shortcut-targets-by-id\0B0xcP-eNvJ9dZDBwVVJOVk5Ld2s/PRJ_SCHELDE/VNSC/Rapportage_INBO/2023/070_macrozoobenthos/figuren/070-figuur-Shannondiv-intert-Zeeschelde} 

}

\caption{Shannon diversiteit per waterlichaam en voor de volledige Zeeschelde voor het intertidaal doorheen de tijd. De shannon diversiteit werd zowel berekend op densiteiten (aantallen/m²) als voor biomassa (g/m²). De Oligochaeta worden niet steeds gedetermineerd en werden als 1 taxon beschouwd.}\label{fig:070-figuur16}
\end{figure}

\begin{figure}[H]

{\centering \includegraphics[width=1\linewidth]{G:/.shortcut-targets-by-id\0B0xcP-eNvJ9dZDBwVVJOVk5Ld2s/PRJ_SCHELDE/VNSC/Rapportage_INBO/2023/070_macrozoobenthos/figuren/070-figuur-Shannondiv-subt-Zeeschelde} 

}

\caption{Shannon diversiteit per waterlichaam en voor de volledige Zeeschelde voor het subtidaal doorheen de tijd. De shannon diversiteit werd zowel berekend op densiteiten (aantallen/m²) als voor biomassa (g/m²). De Oligochaeta worden niet steeds gedetermineerd en werden als 1 taxon beschouwd.}\label{fig:070-figuur17}
\end{figure}

\hypertarget{resultaten-diversiteit-oligochaeta}{%
\subsection{Resultaten diversiteit
Oligochaeta}\label{resultaten-diversiteit-oligochaeta}}

\hypertarget{methode}{%
\subsubsection{Methode}\label{methode}}

Veruit de belangrijkste groep macrozoöbenthos in het intertidaal van de
zoete zones en het Oligochalien van de Zeeschelde is de subklasse van de
Oligochaeta. De determinatie vergt veel tijd en epxertise en is niet
jaarlijks haalbaar. Om toch voeling te houden met de diversiteit en
trends wordt om de 3 jaar een inspanning gedaan om deze groep te
determineren. Daarbij wordt tegenwoordig van een staal een subsample van
50 individuen op naam gebracht. Vaak bevat een staal echter minder dan
50 wormen en worden de x (\textless50) aanwezige wormen gedetermineerd.
Aangezien densiteit en soortenrijkdom doorgaans sterk gerelateerd zijn,
is bij de vorige rapportage van diversiteit Oligochaeta (Van Rijckegem
et al.~2021) gekozen voor een beschrijving en vergelijking van de
gegevens aan de hand van rarefactie. Daarbij worden uit alle verzamelde
individuen random exemplaren getrokken en bij toenemend aantal de
soortenrijkdom berekend. Deze procedure wordt een groot aantal keer
herhaald zodat een vloeiende curve ontstaat. Bij het vergelijken van
verschillende bemonsteringen (jaren, zones van de Zeeschelde,\ldots) kan
slechts vergeleken worden tot het minimum aantal dieren dat in een van
de onderzochte groepen behaald werd. In tegenstelling tot de rapportage
van 2021 werd voor deze rapportage de volledige set van data sinds 2008
omgerekend zodat we een bespreking kunnen doen van de laatste 15 jaar (5
driejaarlijkse monitoringsjaren: 2008-2011-2014-2017-2020). De gegevens
van 2020 waren vorig jaar nog niet beschikbaar en worden daarom dit jaar
gerapporteerd.

\hypertarget{soortenrijkdom-1}{%
\subsubsection{Soortenrijkdom}\label{soortenrijkdom-1}}

Een vergelijking van de soortenrijkdomcurves voor de 5 driejaarlijkse
opnames in de KRWzones van de Zeeschelde staat in de Figuur
\ref{fig:070-figuur18}. Opvallend is dat voor alle jaren de
soortenrijkdom het hoogst is in Zeeschelde I, de zone Zoet kort
verblijf. Dit is eveneens de zone waar we de hoogste densiteiten
oligochaeten vinden (de curve loopt het langst door). Een andere
opvallende vaststelling is dat in het eerste onderzochte jaar 2008 in de
zone Zoet lag verblijf en de zone Oligohalien (Zeeschelde II en III) de
soortenrijkdom hoger lag dan in de daarna volgende jaren. De
soortenrijkdom in de zone Saliniteitsgradiënt (Zeeschelde IV) die
aanvankelijk veel lager ligt dan in de andere zones, is in latere jaren
meer gelijk met deze van zone Zoet lang en Oligochalien, door een kleine
toename van de soortenrijkdom in Saliniteitsgradiënt en een daling in de
andere twee zones. De situatie in 2020 was in lijn met de eerdere jaren,
behalve dat in de zone Zoet lang er een opvallend lage soortenrijkdom
was; zelfs lager dan in Saliniteitsgradiënt.

\begin{figure}[H]

{\centering \includegraphics[width=1\linewidth]{G:/.shortcut-targets-by-id\0B0xcP-eNvJ9dZDBwVVJOVk5Ld2s/PRJ_SCHELDE/VNSC/Rapportage_INBO/2023/070_macrozoobenthos/figuren/ZS1234_rarefy1} 

}

\caption{De relatie tussen intertidale soortenrijkdom en aantal onderzochte Oligochaeta na rarefactie voor de verschillende KRW zones in de Zeeschelde voor verschillende onderzochte jaren: 2008-2011-2014-2017-2020. De curve per KRWzone loopt door tot het totaal aantal gedetermineerde wormen voor die opname.}\label{fig:070-figuur18}
\end{figure}

Een vergelijking doorheen de tijd per KRW zone is eenvoudiger te
beoordelen in de Figuur \ref{fig:070-figuur18b}. Uit deze figuur blijkt
nog duidelijker dat de trend in de diversiteit van Oligochaeta omgekeerd
verloopt in Saliniteitsgradiënt (toename) versus de andere zones (afname
sinds 2008). Vooral in de zones Zoet lang en Oligohalien is de afname na
2008 opvallend. Dit zijn de zones waarin de biologisch te lage
zuurstofconcentratie, als gevolg van organische vervuiling, het sterkst
zijn tol had op het ecologisch functioneren, en er pas na 2007 vissen en
andere grote ongewervelden verschenen. Wel waren er toen enorme
densiteiten Oligochaeta, en blijkbaar ook een grote diversiteit aan
Oligochaeta.

\begin{figure}[H]

{\centering \includegraphics[width=1\linewidth]{G:/.shortcut-targets-by-id\0B0xcP-eNvJ9dZDBwVVJOVk5Ld2s/PRJ_SCHELDE/VNSC/Rapportage_INBO/2023/070_macrozoobenthos/figuren/JaarperZS} 

}

\caption{De relatie tussen intertidale soortenrijkdom en aantal onderzochte Oligochaeta na rarefactie voor de verschillende KRW zones in de Zeeschelde voor verschillende onderzochte jaren: 2008-2011-2014-2017-2020. De curve per jaar loopt door tot het totaal aantal gedetermineerde wormen voor die opname.}\label{fig:070-figuur18b}
\end{figure}

Een gelijkaardige analyse werd gemaakt voor de zijrivieren (Figuur
\ref{fig:070-figuur19}). Voor alle jaren zien we een typisch onderscheid
tussen de iets soortenarmere Durme en Rupel, die een grotere invloed van
de Zeeschelde ondergaan en in overeenstemming daarmee een vergelijkbare
soortenrijkdom hebben, en de meer typische getijrivieren Dijle, Nete en
Zenne die doorgaans een hogere soortenrijkdom hebben. Relatief gezien is
de soortenrijkdom van Rupel en Durme in het laatste monitoringsjaar 2020
lager dan in de voorgaande monitoringsjaren (2008-2011-2014-2017).

\begin{figure}[H]

{\centering \includegraphics[width=1\linewidth]{G:/.shortcut-targets-by-id\0B0xcP-eNvJ9dZDBwVVJOVk5Ld2s/PRJ_SCHELDE/VNSC/Rapportage_INBO/2023/070_macrozoobenthos/figuren/Zijrivier_rarefy} 

}

\caption{De relatie tussen intertidale soortenrijkdom en aantal onderzochte Oligochaeta na rarefactie voor de verschillende KRW zones in de Zeeschelde voor verschillende onderzochte jaren: 2008-2011-2014-2017-2020. De curve per KRWzone loopt door tot het totaal aantal gedetermineerde wormen voor die opname.}\label{fig:070-figuur19}
\end{figure}

\hypertarget{algemene-conclusie}{%
\section{Algemene conclusie}\label{algemene-conclusie}}

\textbf{De soortenrijkdom}

Om de drie jaar wordt de volledige soortenrijkdom (inclusief
determinatie van de Oligochaeta) van het macrozoöbenthos in de
Zeeschelde bepaald, zo ook in 2020. Deze data waren vorig jaar nog niet
beschikbaar, maar worden in dit rapport, samen met de resultaten voor
2021 (zonder determinaties van Oligochaeta), gepresenteerd. De
soortenrijkdom (exclusief Oligochaeta) lijkt de laatste paar jaren licht
te stijgen in verschillende waterlichamen van de Zeeschelde met name de
zone Zoet kort en Saliniteitsgradiënt. In de laatste zone zijn de
laatste jaren een aantal soorten tweekleppigen toegenomen. Voor de
Oligochaeten analyseerden we in deze rapportage voor het eerst de
volledige dataset sinds 2008 aan de hand van rarefactiecurves. Daaruit
blijkt dat er twee types evoluties zijn: in de zoete zones en het
Oligohalien is er sinds 2008 een afname van de soortenrijkdom, in de
zone Saliniteitsgradiënt is er een omgekeerde evolutie naar een
toenemende soortenrijkdom. De zone Zoet kort verblijf bleef de hele
monitoringsperiode wel afgetekend het soortenrijkste waterlichaam. De
resultaten in 2020 waren in lijn met deze van de laatste
monitoringsjaren, met uitzondering van een eerder historisch lage
soortenrijkdom in zone Zoet lang.

\textbf{De Shannon diversiteit}

De Shannon diversiteit is een nieuwe evaluatieparameter die in 2022 voor
het eerst gerapporteerd werd (Van Rijckegem et al.~2022). Het gedrag van
deze parameter en hoe deze best te interpreteren is nog onderhevig aan
voortschrijdend inzicht. Een eerste beoordeling gaf aan dat deze
parameter vrij sterke fluctuaties vertoont en mogelijk minder goed
bruikbaar is in de subtidale zone, door het erratisch verloop overheen
de jaren, mogelijk gelinkt aan lagere densiteiten macrozoöbenthos in
deze zone. De Shannon diversiteit in het intertidaal is in de meeste
zones (behalve sterke saliniteitsgradiënt) heel laag. Dit is te wijten
aan de dominantie van 1 taxon (Oligochaeta). Het verloop vertoont een
toename in vrijwel alle zones sinds 2015. In de zone met sterke
Saliniteitsgradiënt is er recent een duidelijke toename in diversiteit
gebaseerd op aantallen, maar een gelijktijdige afname in diversiteit
gebaseerd op biomassa. Dat laatste fenomeen is wellicht te wijten aan de
opkomst van enkele tweekleppigen (zie verder).

\textbf{De systeembiomassa}

Voor het tweede jaar op rij is de systeembiomassa in 2020 de hoogste
waarde ooit vastgesteld in de recente monitoringcyclus (vanaf 2008) en
overschrijdt ze ruim de doelstelling. De hoge waarde is vooral het
gevolg van een sterke toename van benthosbiomassa in de zone met sterke
Saliniteitsgradiënt. Deze zone heeft veruit de grootste intertidale
fysiotoopoppervlaktes zodat kleine gemiddelde biomassawijzigingen een
grote impact hebben. De veranderende biomassa aan macrozoöbenthos in
deze zone komt voor een groot deel op het conto van een nieuw gevestigde
exotische tweekleppige, de brakwaterkorfschelp. Deze soort is pas in
2018 voor het eerst vastgesteld (Dumoulin \& Langeraert, 2020) maar is
zich in sneltempo aan het uitbreiden, met nu al gevolgen voor
bijvoorbeeld de aantallen overwinterende watervogels (zie hoofdstuk
watervogels). Dit is mogelijks een overschatting omdat tot nu toe vooral
de meest stroomafwaartse zone met sterke saliniteitsgradiënt
gekoloniseerd is. In San Fransisco Bay heeft de invasie door de
brakwaterkorfschelp geleid tot een omslag in het trofisch functioneren
van het ecosysteem. Dichtheden liepen lokaal op tot 10.000/m² en
fytoplankton densiteiten werden sterk gereduceerd (Nichols et al.~1990).
In een dergelijk systeem is de verhouding tussen macrobenthische
biomassa en primaire productiviteit sterk afwijkend van deze in
natuurlijke systemen (Van Hoey et al.~2007). In de Zeeschelde zijn reeds
dichtheden tot 700/m² vastgesteld (Dumoulin \& Langeraert, 2020).

Hoewel de totale systeembiomassa ruim de doelstelling haalt werden de
lokale doelstellingen niet in alle deelzones gehaald. In de Oligohaliene
zone en in de Zoete zone met lang verblijf valt de biomassa
macrozoöbenthos lager uit dan de doelstelling. Dit was overigens in alle
monitoringsjaren tot en met 2020 het geval.

\hypertarget{referenties}{%
\section{Referenties}\label{referenties}}

Dumoulin, E., \& Langeraert W. (2020). De brakwaterkorfschelp
\emph{Potamocorbula amurensis} (Schrenck, 1861) (Bivalvia, Myida,
Corbulidae), een nieuwkomer in het Schelde-estuarium; of het begin van
een lang verhaal. Inleiding. De Strandvlo 40: 113--172.

Nichols F., Thompson J. \& Schemel L. (1990). Remarkable invasion of San
Francisco Bay (California, USA), by the Asian clam \emph{Potamocorbula
amurensis}. II, Displacement of a former community. Marine Ecology
Progress Series 66: 95--101.

Van Hoey G., Drent J. \& Ysebaert T. (2007). The Benthic Ecosystem
Quality Index (BEQI), intercalibration and assessment of Dutch coastal
and transitional waters for the Water Framework Directive - Final
Report. NIOO report 2007-02.

Van Ryckegem G., Vanoverbeke J., Van Braeckel A., Van de Meutter F.,
Mertens W. Mertens A. \& Breine J. (2021). MONEOS-Datarapport INBO:
toestand Zeeschelde 2020. Monitoringsoverzicht en 1ste lijnsrapportage
Geomorfologie, diversiteit Habitats en diversiteit Soorten. Rapporten
van het Instituut voor Natuur- en Bosonderzoek 2021 (47). Instituut voor
Natuur- en Bosonderzoek, Brussel. DOI: doi.org/10.21436/inbor.52484672.

Van Ryckegem G., Vanoverbeke J., Van Braeckel A., Van de Meutter F.,
Mertens W. Mertens A. \& Breine J. (2021). MONEOS-Datarapport INBO:
toestand Zeeschelde 2020. Monitoringsoverzicht en 1ste lijnsrapportage
Geomorfologie, diversiteit Habitats en diversiteit Soorten. Rapporten
van het Instituut voor Natuur- en Bosonderzoek 2021 (47). Instituut voor
Natuur- en Bosonderzoek, Brussel. DOI: doi.org/10.21436/inbor.52484672.

\end{document}
